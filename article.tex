\documentclass[twoside,letterpaper,twocolumn]{article}

%%%%%%%%%%%%%%%%%%%%%%%%%%%%%%%%%%%%%%%%%%%%%%%%%%%%%%%%%%%%%%%%%%%%%%
% Package with format specifications for the 16th CISBGf
\usepackage{cisbgf19}

%%%%%%%%%%%%%%%%%%%%%%%%%%%%%%%%%%%%%%%%%%%%%%%%%%%%%%%%%%%%%%%%%%%%%%
% MANDATORY PARAMETERS

% Setting the title
\title{Latex Template for International Congress of the Brazilian Geophysical Society}

% Setting the authors
\author{Felipe Timóteo da Costa, Adelson Oliveira, Jurandyr Schmidt, Adelson Oliveira, Rodrigo Portugal,Hedison Kiuity Sato}

% Setting the headings
\headauthor{Costa, Felipe Timóteo}
\headtitle{Latex Template}

%%%%%%%%%%%%%%%%%%%%%%%%%%%%%%%%%%%%%%%%%%%%%%%%%%%%%%%%%%%%%%%%%%%%%%
%%%%%%%%%%%%%%%%%%%%%%%%%%%%%%%%%%%%%%%%%%%%%%%%%%%%%%%%%%%%%%%%%%%%%%
\begin{document}

\maketitle

\begin{abstract}

\lipsum[1]

\end{abstract}

\section{Introduction}

\lipsum[2]

\begin{figure}[h!]
	\centering
	\subfloat[]{
		\includegraphics[width=0.45\linewidth]{images/logo_congress}
		\label{fig:label1}
	}
	\subfloat[]{
		\includegraphics[width=0.45\linewidth]{images/logo_congress}
		\label{fig:label2}
	}
	\caption{Example using figures side by side a) Figure on the left. b) Figure on the right. }
	\label{fig:label3}
\end{figure}

\lipsum[6]

\section{Materials and Methods}

\subsection{making a citation}

First, you need a bibliography file. Google it to discover how to do it. Tips: The Mendeley program could provide a good solution.

I save the file with the name references.bib in this folder.

Example of indirect citation \citep{Yilmaz2000}.

The book of professor \cite{Claerbout1984} is a good choice for who want to learn more about seismic processing and imaging. This was an example of the direct citation.


\subsection{Optional subsection}
\lipsum[7]

The 2D acoustic wave equation:
\begin{equation}\label{waveequation}
\frac{ 1}{  c^2(\mathbf{x})}\frac{\partial^2}{\partial t^2}p(\mathbf{x},t) - \nabla^2 p(\mathbf{x},t)  = w(\mathbf{x}_s,t),				
\end{equation}

\noindent
\lipsum[8]

\begin{figure}[h!]
	\centering
	\includegraphics[width=0.9\linewidth]{images/logo_congress}
	\label{fig:label4}	
	\caption{Example using a simple figure}
\end{figure}

\lipsum[8]
Example of an equation using 2 lines:

\begin{eqnarray}\label{unconstrainedLagrangian}
 \begin{array}{rl}
\mathcal{L}(w,p,p^{\dagger}) = &\displaystyle \frac{1}{2}\int_{T} \| \mathbf{d}_{obs}(z_{rec},t) - \mathbf{d}_{cal}(z_{rec},t) \|^2 dt \\
                               &\displaystyle - \int_{T}p^{\dagger}\left[ F\left(p(z,t)\right) - w(t) \right] dt ,
\end{array}   
\end{eqnarray}

\noindent
dummy text.

\lipsum[9]
\section{Results}

\subsection{Optional subsection}

\lipsum[10]

\begin{figure}[h!]
	\centering
	\subfloat[]{
		\includegraphics[width=0.45\linewidth]{images/logo_congress}
		\label{fig:label5}
	}
	\subfloat[]{
		\includegraphics[width=0.45\linewidth]{images/logo_congress}
		\label{fig:label6}
	}\\
	\subfloat[]{
		\includegraphics[width=0.45\linewidth]{images/logo_congress}
		\label{fig:label7}
	}
	\subfloat[]{
		\includegraphics[width=0.45\linewidth]{images/logo_congress}
		\label{fig:label8}
	}
	\caption{ Example using 4 figures at once. a) Above on the left b)Above on the right  c)Bellow on the left d) Bellow on the right.}
	\label{fig:label9}
\end{figure}

\lipsum[15]






\begin{table}[htb]
	\caption{Descrição tabela.}
	\label{tab:label1}
	\begin{tabular}{|c|c|c|c|}
		\hline 
		\textbf{Methods} & \textbf{column 1} & \textbf{column 2} & \textbf{column 3} \\ 
		\hline 
		Seismological    &  &  &  \\ 
		\hline 
		Gravitational    &  &  &  \\ 
		\hline 
		Electric         &  &  &  \\ 
		\hline 
		Magnetic         &  &  &  \\ 
		\hline 
		Electromagnetic  &  &  &  \\ 
		\hline 
		Thermal          &  &  &  \\ 
		\hline 
	\end{tabular} 
\end{table}


\section*{Discussion and Conclusion}

\lipsum[1-3]


\bibliography{references}

%%% Uncomment if you don't have references bib file
%\section{References}

%Barsh, R. (2009) Web-based technology for storage and pro- cessing of multi-component data in seismology. First steps towards a new design, PhD thesis, Ludwig Maximilians Universitat Munchen, Munich, Germany, 126 pp.
%
%Percivall, G. (2010). The application of open standards to en- hance the interoperability of geoscience information, In- ternational Journal of Digital Earth, 3 (S1), 14-30.
%
%Pirchiner, M ; Collaço, B. B. ; Calhau, J. ; Assumpção, M. S. ; Dourado, J. C. . BRAzilian Seismographic Integrated Systems (BRASIS): infrastructure and data management. Annals of Geophysics, v. 54, p. 17-22, 2011.
%
%Schorlemmer, D., A. Wyss, S. Maraini, S. Wiemer and M. Baer (2004). QuakeML - an XML schema for seismology, ORFEUS Newsletter, 6 (2); http://www.orfeus-eu.org/Organiza- tion/Newsletter/vol6no2/quakeml.shtml (last accessed 7 March 2011).
%
%SEED Manual (2010). SEED Reference Manual, SEED Format Version 2.4, May 2010, IRIS, 212 pp.; 
%
%SeisComP3.org (2011). SeisComP 3 documentation; www.seiscomp3.org/wiki/doc (last accessed 20 April 2013).

\section{Acknowledgments}

\lipsum[11]

\end{document}


